\documentclass[]{article}

\usepackage{amsmath,amsthm,amssymb}
\usepackage{multicol}

\renewcommand{\qedsymbol}{$\blacksquare$}

\begin{document}

\title{CS 161 - Assignment 7}
\author{Lowell Bander}
\maketitle

\begin{enumerate}
\item \begin{enumerate}
\item \begin{proof}
\begin{align}
P(A,B|K) &= \frac{P(A,B,K)}{P(K)} \\
&= \frac{P(A|B,K)P(B,K)}{P(K)}\\
&= \frac{P(A|B,K)P(B|K)P(K)}{P(K)}\\
&= P(A|B,K)P(B|K)\\
\end{align} \end{proof}
\item \begin{proof}
\begin{align}
P(A|B,K) &= \frac{P(B,A,K)}{P(B,K)}\\
&= \frac{P(B|A,K)P(A,K)}{P(B,K)}\\
&= \frac{P(B|A,K)P(A|K)P(K)}{P(B,K)}\\
&= \frac{P(B|A,K)P(A|K)P(K)}{P(B|K)P(K)}\\
&= \frac{P(B|A,K)P(A|K)}{P(B|K)}\\
\end{align}
\end{proof}
\end{enumerate}
\newpage
\item If we assign $O$ to the state of oil being present, $G$ to natural gas being present, and the test result coming back positive to $T$, we have the following probabilities:
\begin{multicols}{2}
$P(O)=0.5$\\
$P(G \land \neg O)=0.2$\\
$P(\neg G \land \neg O)=0.3$\\
$P(T|O)=0.9$\\
$P(T|G\land \neg O)=0.3$\\
$P(T|\neg G\land\neg O)=0.1$
\end{multicols}

The probability of oil being present given that the test comes back positive can be written as
\begin{align}
P(O|T) &= \frac{P(T|O)P(O)}{P(T)}\\
&= \frac{P(T|O)P(O)}{P(T|O)P(O)+P(T|G\land\neg O)P(G\land\neg O) + P(T|\neg G\land\neg O)P(\neg G\land\neg O)}\\
&= \frac{0.6\times 0.5}{0.6\times 0.5 + 0.3\times 0.2 + 0.1\times0.3}\\
&= 83.\overline{33}\%
\end{align}
\item The condition of the random object being square is denoted by $S$, with circle being denoted by it's negation. Similarly, $O$ and $Y$ denote the condition of the object being labeled with a 1 and being the color yellow, respectively, with the negations denoting 2 and blue, respectively. \\

\begin{table}[h]
\centering \begin{tabular}{c | c c c | c}
$w_i$ & $S$ & $O$ & $Y$ & $P(\cdot)$\\\hline
1 & T & T & T & 2/13 \\
2 & T & T & F & 1/13 \\
3 & T & F & T & 4/13 \\
4 & T & F & F & 1/13 \\
5 & F & T & T & 1/13 \\
6 & F & T & F & 1/13 \\
7 & F & F & T & 2/13 \\
8 & F & F & F & 1/13 \\
\end{tabular}
\end{table}
Therefore, $P(\alpha_1)=P(w_1)+P(w_3)+P(w_5)+P(w_7)=9/13$. Similarly, $P(\alpha_2)=P(w_1)+P(w_2)+P(w_3)+P(w_4)=8/13$.
\newpage
\item \begin{enumerate}
\item The Markovian assumptions asserted by the given directed acyclic graph are as follows.
\begin{multicols}{2}
\begin{itemize}
\item $I(A,\emptyset,BE)$
\item $I(B,\emptyset,AC)$
\item $I(C,A,BDE)$
\item $I(D,AB,CE)$
\item $I(E,B,ACDFG)$
\item $I(F,CD,ABE)$
\item $I(G,F,ABCDEH)$
\item $I(H,FE,ABCDG)$
\end{itemize}
\end{multicols}
\item \begin{itemize}
\item False, because the path $A\rightarrow D \rightarrow F \rightarrow H \rightarrow E$ is not blocked by valve $H$, because it is a a convergent valve and is contained in the knowledge base.
\item False, because the path $G \rightarrow F \rightarrow C \rightarrow A \rightarrow D \rightarrow B \rightarrow E$ is not blocked by valve $D$ because it is a convergent valve and is contained in the knowledge base.
\item False, because there exists the unblocked path $B \rightarrow E \rightarrow H$.
\end{itemize}
\item 
\begin{align}
P(a,b,c,d,e,f,g,h)&=P(a|b,c,d,e,f,g,h)\\
&\times P(b|c,d,e,f,g,h)\\
&\times P(c|d,e,f,g,h)\\
&\times P(d|e,f,g,h)\\ 
&\times P(e|f,g,h)\\
&\times P(f|g,h)\\ 
&\times P(g|h)\\
&\times P(h)
\end{align}

\item \begin{itemize}
\item $P(A=0,B=0)=P(A=0)P(B=0)=0.24$, because the joint probability is equal to the product of the marginals for independent variables.
\item Because $A$ and $E$ are independent, $P(E=1|A=1)=P(E=1)$, which expands to $P(E=1)=P(E=1|B=0)P(B=0)P(E=1|B=1)P(B=1)=0.9\times 0.3+ 0.1\times 0.7=0.34$
\end{itemize}
\end{enumerate}
\end{enumerate}

\end{document}
